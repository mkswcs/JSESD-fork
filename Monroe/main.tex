\documentclass[11pt]{sig-alternate}
\usepackage{hyperref}
\usepackage{tabularx}
\usepackage{graphicx}
\usepackage{blindtext}
\usepackage[utf8]{inputenc}
\usepackage[english]{babel}
\usepackage{lastpage}
\usepackage{comment}
\usepackage{dirtytalk}

\usepackage[backend=biber, style=apa]{biblatex}
\addbibresource{notation.bib}
\usepackage{authblk}

\usepackage{fancyhdr}
\pagestyle{fancy}
\renewcommand{\headrulewidth}{0pt}
\renewcommand{\footrulewidth}{0pt}
\setlength\headheight{80.0pt}
\addtolength{\textheight}{-80.0pt}
\chead{%
  \ifcase\value{page}
  % empty test for page = 0
  \or \includegraphics[width=\textwidth]{headerimage1.png}% page=1
  \or \includegraphics[width=\textwidth]{headerimage1.png}% page = 2
  \or \includegraphics[width=\textwidth]{headerimage1.png}% page = 3
  \or \includegraphics[width=\textwidth]{headerimage1.png}% page = 4
  \or \includegraphics[width=\textwidth]{headerimage1.png}% page = 5
  \else
  \includegraphics[width=\textwidth]{headerImage1.png}
  \fi
}
%\chead{\includegraphics[width=\textwidth]{headerImage.png}}
\fancyfoot[LE,LO]{Overview of the proceedings of the 2021 Inclusion in Science, Learning a New Direction, Conference on Disability\\           
DOI: 10.14448/jsesd.13.0002}
\fancyfoot[CE,CO]{{ }}
\fancyfoot[RE,RO]{\thepage}
\pagenumbering{arabic}
\hypersetup{
    colorlinks=true,
    urlcolor=blue
}
 
\let\oldabstract\abstract
\let\oldendabstract\endabstract
\makeatletter
\renewenvironment{abstract}
{\renewenvironment{quotation}%
               {\list{}{\addtolength{\leftmargin}{1em} % change this value to add or remove length to the the default
                        \listparindent 1.5em%
                        \itemindent    \listparindent%
                        \rightmargin   \leftmargin%
                        \parsep        \z@ \@plus\p@}%
                \item\relax}%
               {\endlist}%
\oldabstract}
{\oldendabstract}
\makeatother
\begin{document}


\title{Overview of the proceedings of the 2021 Inclusion in Science, Learning a New Direction, Conference on Disability }

\author[1]{\large Cary A. Supalo}
\author[2]{Jasodhara Bhattacharya}
\author[2]{Daniel Steinberg}
\affil[1]{Educational Testing Service}
\affil[2]{Center for Complex Materials, Princeton University}
\end{large}
\toappear{}

\maketitle
\begin{@twocolumnfalse} 

\begin{abstract}
\begin{large}
     \textit {The 12th annual Inclusion in Science, Learning A New Direction, Conference on Disability (ISLAND) was held virtually at Princeton University on Friday, September 17 and Saturday, September 18, 2021. This year’s proceedings consisted of thirteen different presentations.}
\end{large}
\end{abstract}
\end{@twocolumnfalse}

%% ABSTRACT
\begin{tabular}{ p{16cm}}
%% AUTHOR INFORMATION
\\ \\ \\
\textbf{*Corresponding Author, Cary A. Supalo} \href{mailto:csupalo@ets.org}{(csupalo@ets.org)} \\
\textit{Submitted March 15, 2022 }\\
\textit{Accepted April 20, 2022} \\
\textit{Published online July 15, 2022} \\
\textit{DOI: 10.14448/jsesd.14.0002} \\
\end{tabular}

\pagebreak
\vspace{5mm}
\section*{\vspace{140mm}}
\large
\vspace{2.5 mm}\\
{The first presentation was delivered by Dr. Daniel Steinberg from Princeton University. His presentation is titled, “Building new partnerships in the science access education community and how people with disabilities play a key role in that process. Improving STEM Research Outreach Efforts Working with Communities with Blindness and Vision Impairment (BVI).” This presentation discussed several of the different partnerships the Princeton Center for Complex Materials (PCCM) has already established. These programs include STEM on Stage, Princeton University Materials Academy, and Dia de la Ciencia bilingual Day of Science. There is an interest at Princeton University to incorporate students with visual impairments into research experiences for undergraduate students or (REU) programs. These students would work with Princeton University faculty as part of a summer research experience to learn how to perform innovative science research. Additionally, there are programs for K12 teachers to conduct similar research and can be first or second author on research papers based on their work. Overall, leveraging the partnerships that are available through the Materials Research Science and Engineering Centers (MSREC) network sponsored by the National Science Foundation at universities across the United States to work with and be involved with educational outreach programs can and do serve as conduits for people with disabilities interested in science to be involved in innovative scientific research.}
\vspace{2.5 mm}\\Next, Dr. Siggy Schmid from the University of Sydney located in Sydney, Australia presented “Developing laboratory procedures in chemistry for independent learning of students who are blind or low vision.” This presentation gave an overview of recent chemistry laboratory access methodologies and techniques that appear in chemical education literature. This presentation also discussed a newly funded project to make the general chemistry laboratory curriculum more accessible to blind and low vision students in a firsthand way. This effort involves the use of text-to-speech enabled laboratory equipment along with methodological adaptations to provide access to color change and other procedural visual information. It is the expectation that these theoretical chemistry laboratory procedures will be field tested with a blind student in 2022. Once feedback is collected, a set of final modifications will be developed prior to this curriculum being made available to any students with visual impairments who may enroll at the University of Sydney. Chemistry laboratory courses serve as a gateway course to numerous other STEM related fields of study and with the recent commitments by the University of Sydney to promote inclusion in all academic programming, this effort parallels that commitment in a very synergistic way.
\vspace{2.5 mm}\\The final presentation on the opening day of the conference was delivered by Annalise Diodato and was titled, “Learning to be a blind Scientist: Accessible Science from an Autoethnographic Perspective”. This presentation was an auto-ethnographic perspective on her walk with learning that she is blind and her passion for pursuing employment in a STEM career and the challenges associated with that goal. Her presentation described her transition from acceptance of a vision impairment to learning the necessary blindness skills to be a more competent blind scientist. Learning how to perform data collection and analysis non-visually. Meeting other blind scientists also served as an inspirational support on her walk with science. Annalise indicated that she has a strong desire to work in a STEM laboratory field and is doing what must be done to gain the successful skills and self-confidence in herself to accomplish that goal.
\vspace{2.5 mm}\\On the second day of the conference, there were ten additional presentations. The first of which was titled, “Teaching Microscopic Concepts in Chemistry to Neurodiverse Students,” delivered by Dr. Christa Monroe from Landmark College. This presentation discussed methodologies for teaching microscopic concepts to diverse learners. This includes students with various learning disabilities at Landmark College. Executive functional challenges can carry based on the student’s individualized needs. She discussed how to leverage universal design principals in her teaching and instruction in both low and high-risk assignments. Giving students options to complete their assignments from oral to written discussions using multi-modal approaches to communicating chemical concepts.
\vspace{2.5 mm}\\Teaching microscopic concepts is challenging because students can not directly see the science principals in question. This is particularly true with teaching atomic structure and making prediction about chemical behavior. Teaching students how to apply models to specific scenarios of chemical concepts. These models can serve as physical manipulatives to interact and learn microscopic concepts. Overall, these physical manipulatives will be expanded into other subject areas of chemistry to see how students who are classified as Nero diverse.
\vspace{2.5 mm}\\This presentation was followed by Dr. Todd Pagano from Rochester Institute of Technology (RIT) entitled, “Sustaining an Open Access Journal focused on Educating Students with Disabilities in STEM.” This presentation discussed ways to sustain an open access journal. JSESD disseminates science education concepts as it pertains to students with disabilities. One of the goals of JSESD is to provide equal access to the JSESD journal to readers. Open@RIT is a center at RIT that promotes open access and sharing of products. There is a core team of RIT staff for the purposes of enhancing visibility and translation of Open Access content generated at RIT and shared with the world. Open@RIT is assisting with PDF conversation of articles into a more accessible version to provide equal and universal access. The decision to use the HTML file format was made for the publication of journal articles. LaTex was employed to help create a more accessible HTML format of articles. This process/script is still under development so that it can be interpreted by text-to-speech screen readers. It is the hope of the leadership of the journal to go back into past issues of JSESD and work to convert these older issues to more accessible HTML versions in the future. The goal is that this methodology discussed as part of this presentation will lead to more inclusive and accessible publication practices for all open access journals.
\vspace{2.5 mm}\\The next presentation was delivered by Dr. Bryan Shaw from Baylor University and is titled, “Making Molecular imagery more accessible to students with blindness.” This presentation discussed the use of micro-models to convey biochemical protein structure information to a sample of blind folded sighted students. It was found that approximately 85\% of the micro-models were accurately identified by either or both finger and tung investigations. These micro-models are smaller than traditional models thus lowering the price and increasing the larger number of molecular structures that can be 3D printed and/or produced using gummy or a silicon resin. It is felt this methodology can be used to make larger numbers of 3D molecular structures for blind students thus making biochemical investigations more accessible.
\vspace{2.5 mm}\\Followed by Dr. Tim Spuck from Associated Universities Inc and was titled, “Building Tools for Inclusive Astronomy: IDATA and Other AUI Initiatives.” This presentation discussed the Afterglow Access web-based data visualization tool for astronomy that is free for public use and does not require any software installation on a host computer. This work was funded by the National Science Foundation. Audio sonification tools were developed that used both volume and pitch using a piano keyboard as the basis for the audio output. The increase in volume represents brightness and an increase in tone represents where you are on the picture of the dataset. Everything in a virtual line has the same tone. A user can change the time viewing of the picture to get a more detailed view. For more information, go to \href{https://www.idataproject.org}{idataproject.org}.
\vspace{2.5 mm}\\Next up was Dr. Greg Williams who delivered a presentation titled, “Accessible Audio Periodic Table.” This presentation discussed a new web-based audio periodic table that leverages sonification tools to communicate periodic trend information. It is this periodic trend data that is normally communicated through color and/or drawn arrows on a periodic table. A blind screen reader user can access a dialog box on this web based periodic table to set the specific periodic trend of interest. A keystroke can be initiated to play the trend based on where the cursor is located on the periodic table. The Accessible Audio Periodic Table (AAPT) is available free on the Independence Science website. The conference then took a break for lunch.
\vspace{2.5 mm}\\The first presentation after the lunch break was delivered by Dr. Jason White entitled, “Using Markup Languages for Accessible Scientific, Technical, and Scholarly Document Creation” This presentation discussed the use of markup languages such as HTML, LaTeX, Markdown, or ASCII may be used in place of word processors to increase accessibility of documents for both author and reader. This would include the ability to use graphical content. These documents are easily converted between different file formats thus making accessibility of these documents an enhanced experience.
\vspace{2.5 mm}\\This presentation was followed by Dr. Tiffany Wild who spoke about, “Nemeth in a Box.” This work was funded by the United States Department of Education, and it seeks to help both teachers and students further understand properties of the Nemeth braille code. Boxes containing puzzles and hands-on learning experiences were used to assess student knowledge of the Nemeth braille code. This would then determine the impact of problem-based learning self-efficacy levels of participants before and after the Nemeth in a Box program and the overall project impact on student knowledge of the Nemeth code symbols. Some of the key outcomes showed benefits of firsthand learning for grades 6-8 students who were learning the Nemeth Braille code and enhanced levels of proficiency during the Covid19 Pandemic. The success of this program will lead to projects that use this approach in future Nemeth code braille instruction.
\vspace{2.5 mm}\\The next speaker was Robert Jaquiss who delivered a presentation titled, “Techniques for B/VI students enrolled in coding courses.” This presentation discussed many of the challenges blind and vision impaired students face who take an interest in computer coding. Limitations of graphical user interfaces and different command line options that have proven themselves to be very useful and valuable work arounds for blind people in the computer coding space. Additionally, the concept of drag and drop coding is visual code that is designed for people with less experience with programming to engage in computer coding. However, it is difficult if not impossible for blind computer coders to code in this fashion. Change in current computer coding methodology is necessary in order to ensure blind computer coders will continue to have access to this lucrative profession.
\vspace{2.5 mm}\\After the afternoon break, the final two presentations delt with mentorship programs in STEM for students with disabilities. The first presentation was given by Kevin Fjelsted “STEM-torship a new 501c3 for blind science mentorship” from STEM Mentorship and Ashley Neybert. This new program leverages a triangular approach that consists of a blind mentor, a scientist, and an instructor or teacher to provide support for the blind student. This program seeks to recruit volunteers in all three of these categories to work with the first group of blind students. This 501C3 organization seeks to raise funds to start the first recruitment campaign to start providing support for the next generation of blind students initially in chemistry and later other physical sciences. This program views all participants as mentors that are working together in a synergistic way to provide the necessary supports needed for a successful inclusive hands-on science learning experience.
\vspace{2.5 mm}\\Last but certainly not least was by Logan Gin and was titled, “Students with Disabilities in Undergraduate Research, Challenges and Possibilities.” This presentation consisted of the results of a survey that consisted of both quantitative and qualitative data from a sample of over 100 self-identified students with disabilities who were enrolled in life science undergraduate courses. Twenty students who participated in the initial survey agreed to be interviewed for more holistic details on their undergraduate research experience. This research indicates that most of the accommodations provided to students with disabilities were done so without student support services involvement. Rather, most of the accommodations were provided between undergraduate student with a disability and their research advisor. Further, many of the types of accommodations that were devised were developed by the student with a disability based on their individualized needs. The survey data indicated that the students with disabilities provided different perspectives on research teams consisting of graduate students and faculty. Overall, this research indicates that undergraduate research experiences for students with disabilities are possible and can have a significant impact on whether or not a student with a disability decides to pursue a STEM career path by pursuing an advanced degree in a STEM field.
\vspace{2.5 mm}\\This concludes the 2021 ISLAND conference summary of presentations. We hope that readers will plan to attend the 2022 ISLAND conference to be held over the weekend of September 16-17, 2022, at Princeton University. The presentations and activities will provide a snapshot of what is going on in the field of science accessibility for students with disabilities. Together we can all work together to move the field forward in promoting a more inclusive science education learning experience. Through education serving as a transformative vehicle for positive change, persons with disabilities will make significant contributions in the Science, Technology, Engineering, and Mathematics professions. 


\end{document}
